\documentclass{sintefdoc}

\usepackage{sintefcolor,hyperref}

\title{The \texttt{sintefcolor} Package}
\author{Federico Zenith}

\newcommand{\testcolor}[1]{\colorbox{#1}{\textcolor{#1}{test}} #1}

\begin{document}

\maketitle

\section*{The SINTEF Corporate Colours}
This package defines SINTEF's corporate colour palette by means of the
\texttt{xcolor} package.
The colours can be used with all the commands of the \texttt{xcolor} package,
refer to \href{http://mirror.ctan.org/macros/latex/contrib/xcolor/xcolor.pdf}
{its documentation}.

All colours are defined in the HTML colour model of \texttt{xcolor} as indicated
in the SINTEF profile manual.

\subsection*{Main Colour}
The main colour is \texttt{sintefblue}, which should be in the foreground and
dominating; it is available in 9 standard shades, from full colour towards white,
and should be used about 60\,\% of the time.
To obtain shades, use \texttt{xcolor}'s standard blending notation as follows:
\begin{itemize}
\item \testcolor{sintefblue}
\item \testcolor{sintefblue!90!white}
\item \testcolor{sintefblue!80!white}
\item \testcolor{sintefblue!70!white}
\item \testcolor{sintefblue!60!white}
\item \testcolor{sintefblue!50!white}
\item \testcolor{sintefblue!40!white}
\item \testcolor{sintefblue!30!white}
\item \testcolor{sintefblue!20!white}
\item \testcolor{sintefblue!10!white}
\end{itemize}
The profile manual presents these shades, but does not maintain they are
exhaustive---only that \texttt{sintefblue} can be shaded.

\subsection*{Contrast Colours}
Contrast colours are used for graphical items and for highlighting. They are not
supposed to be shaded, and each should be used about 8\,\% of the time.
\begin{itemize}
\item \testcolor{sintefcyan}
\item \testcolor{sintefmagenta}
\item \testcolor{sintefgreen}
\item \testcolor{sintefyellow}
\end{itemize}

\subsection*{Additional Colours}
Additional colours are shades of grey, to be used in backgrounds and neutral
elements. They come in 5 shades towards white, and each should be used about
4\,\% of the time.
\begin{itemize}
\item \testcolor{sintefgrey}
\item \testcolor{sintefgrey!80!white}
\item \testcolor{sintefgrey!60!white}
\item \testcolor{sintefgrey!40!white}
\item \testcolor{sintefgrey!20!white}
\item \testcolor{sinteflightgrey}
\item \testcolor{sinteflightgrey!80!white}
\item \testcolor{sinteflightgrey!60!white}
\item \testcolor{sinteflightgrey!40!white}
\item \testcolor{sinteflightgrey!20!white}
\end{itemize}
Again, the profile manual presents these shades, but does not maintain they are
exhaustive.
Furthermore, both \texttt{*grey} colours have a corresponding \texttt{*gray}
identical variant.

\end{document}
