\documentclass[unrestricted]{sintefstatus}

\title{SIN\TeX}
\client{SINTEF's \LaTeX\ users}
\clientref{B0U771M3}
\statusdate{31st July, 2011}
\project{SIN\TeX}
\lastexp{1 NOK}
\totalexp{2 NOK}
\wages{3 NOK}
\planexp{4 NOK}
\lasthrs{5}
\totalhrs{6}
\academic{Yes}
\onschedule{Yes}
\onbudget{Yes}
\manager{Federico Zenith}
\approved{Hopefully you users $\ddot\smile$}

\plannedexpenditures{
              (2011-03-01, 10)
              (2011-04-01, 20)
              (2011-05-01, 30)
              (2011-06-01, 40)
              (2011-07-01, 50)
              (2011-08-01, 70)
              (2011-09-01, 75)
              (2011-10-01, 80) }
\accumulatedexpenditures{
              (2011-03-01,  8)
              (2011-04-01, 13)
              (2011-05-01, 29)
              (2011-06-01, 44)
              (2011-07-01, 58) }

\begin{document}
\frontmatter

\section{Minimum example}
\begin{verbatim}
\documentclass{sintefstatus}
\begin{document}
\frontmatter
Hello world!
\end{document}
\end{verbatim}
Use \verb|\frontmatter| to set up the first page.

\section{Progress in the last period}
This new \LaTeX\ class for SINTEF status reports was produced.
This is also supposed to be compiled with Xe\LaTeX\ or Lua\LaTeX, due to the
presence of SINTEF font specifications.

Most commands to set the fields in the start page will be shown in the PDF when
compiling the document without them, so just compile and read if you are not sure.

\section{Deviations from the plan}
The only non-self-explaining commands are those for the budget plots.
In order to set the data, you need to use the \verb|\plannedexpenditures| and
\verb|\accumulatedexpenditures| commands.
These take a list of coordinates ($x$ for dates, $y$ for \emph{thousands} of
crowns\footnote{The original Word template used plain crowns, but it looks ugly
and silly to have so many \texttt{000} on the ordinate axis, plus it is not very
readable. I will change it back the day someone actually comes to me and proves
me they have a project with a budget of less than 1000\,crowns.}).
The data is input as follows:
\begin{verbatim}
\plannedexpenditures{
        (2011-03-01, 10)
        (2011-04-01, 20)
        (2011-05-01, 30)
        (2011-06-01, 40)
        (2011-07-01, 50)
        (2011-08-01, 70)
        (2011-09-01, 75)
        (2011-10-01, 80) }
\accumulatedexpenditures{
            (2011-03-01,  8)
            (2011-04-01, 13)
            (2011-05-01, 29)
            (2011-06-01, 44)
            (2011-07-01, 58) }
\end{verbatim}
There is nothing wrong with putting the data points on the same line inside the
braces, though I prefer them stacked so I can look them up more easily.

If you need to report your budget progress in another currency, you can use the
\verb|\currency| command; typically, you will type \verb|\currency{€}|.

\section{Plan for the next period}
Please report any errors, improvements or suggestions either to me directly or
on eRoom.

\end{document}
