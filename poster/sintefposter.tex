\documentclass{sintefposter}

\usepackage{hyperref,multicol,wrapfig}

\title{The \texttt{sintefposter} class}

\author{Federico Zenith}

\institute{SINTEF Applied Cybernetics, Trondheim, Norway}

\email{federico.zenith@sintef.no}

\conference{SIN\TeX project}

\begin{document}
\maketitle

\begin{multicols}{3}
\section*{Class Description}
This class is meant to prepare posters for academic conferences.
The class is not particularly original: it is essentially a custom configuration
of \texttt{sciposter}.

The SINTEF corporate colors are taken from the \texttt{sintefcolor} package,
which is therefore necessary to have together with this class.
Because of the requirement of using the Calibri font, \texttt{sintefposter} must
be compiled with Xe\LaTeX\ or Lua\LaTeX.

The poster title is set to be in SINTEF blue and \verb|\huge| size.

All class options will be passed to \texttt{sciposter}, so you should refer to
its documentation on CTAN, however some sensible defaults were chosen for you:
\begin{itemize}
  \item Font size is set to \texttt{36pt}. The default for \texttt{sciposter} is
        ludicrously small, unless you have bad results and want no one to read
        them.
  \item Sections are a bit more sober than the default in \texttt{sciposter}
        (\texttt{plainsections} option).
        If you choose to use the block layout, the colours have already been
        changed to SINTEF corporate colors.
  \item The SINTEF logo automatically appears in the left corner.
        This can be overridden with \verb|\leftlogo{}| or \verb|\noleftlogo|.
\end{itemize}

\section*{Suggestions \& Links}
Refer to
\href{http://mirrors.ctan.org/macros/latex/contrib/sciposter/scipostermanual.pdf}
{\texttt{sciposter}'s manual} for how to customise the poster layout.

To insert figures with text wrapped around them, use
\href{http://mirrors.ctan.org/macros/latex/contrib/wrapfig/wrapfig-doc.pdf}{the
\texttt{wrapfig} package}.

For best results in plots, the
\href{http://mirrors.ctan.org/graphics/pgf/contrib/pgfplots/doc/latex/pgfplots/pgfplots.pdf}
{\texttt{pgfplots} package} makes it easy to keep the same font style and size
in plots as in the text.

\end{multicols}

\end{document}

