\documentclass{sintefdoc}

\title{The \texttt{sintefdoc} class}
\author{Federico Zenith}

\begin{document}
\maketitle

The \texttt{sintefdoc} class is a base class that was not originally intended
to be used by itself, rather it contains a number of commonly used options,
commands and items to be used in subclasses.

\section{Class Options}
The class is a subclass of the standard \texttt{article} class, and takes all
its options with the exception of \texttt{twocolumn}, which is ignored.

The default language is British English, which can be explicitly set with
options \texttt{english} or \texttt{engelsk}; Norwegian can similarly be set
with \texttt{norwegian} or \texttt{norsk}.
The language influences both some labels and the hyphenation patterns,
by means of the \texttt{polyglossia} package.

The classification of the document can be set with
\texttt{unrestricted}, \texttt{internal} (default), \texttt{restricted} and
\texttt{confidential}.
Some subclasses do not make use of this option, for example
\texttt{sintefletter}.

If you want to use digital signatures in SIPOK, you should set the
\texttt{digital} option; manual signatures are the default, and can be
forced with the \texttt{manual} option.
Digital signatures may become the default at a later point.

\section{The Configuration File}
User custom data is stored in a particular file, \texttt{sintefconfig.tex}.
Since this data is always the same for a given user, it is tedious to reinsert
it over and over.
The file contains data such as mailing and visiting addresses, phone numbers,
e-mail and so on; customise the file to your needs and place it beside
\texttt{sintefdoc.cls}.

\section{Dependencies}
The class depends on the \texttt{sintefcolor} package and on some other packages
found in most modern \TeX\ distributions, such as \TeX\ Live or MiK\TeX.

Instead of Times New Roman, the package uses it clone XITS,
from the \texttt{xits} \TeX\ package. The package tries first to find the Calibri
font in the system, which this should work on all machines with Office 2007 or
later. If Calibri is not found, as it can be the case on Linux machines, the
package tries a compatible font called Carlito.
Note that compilation will take longer without Calibri, as Xe\LaTeX will spend some
time trying to find it.

\section{Compilation}
Since SINTEF templates mandate specific TTF fonts (e.g. Calibri and Times New
Roman), documents must be compiled with Xe\LaTeX\ or Lua\LaTeX, not \LaTeX\ or
PDF\LaTeX.
These more modern compilers produce directly a PDF, otherwise they works just
like \LaTeX.

\begin{verbatim}
$ xelatex document.tex
$ lualatex document.tex
\end{verbatim}

\end{document}
